 \documentclass[12pt,letterpaper]{article}
 \usepackage[utf8]{inputenc}
 \usepackage{lmodern}
 \usepackage{tcolorbox}
 \usepackage[margin=0.5in]{geometry}
 
 
 \title{Script}
 \newcounter{example}[enumi]
 \setcounter{example}{0}
 \stepcounter{example}
\begin{document}
	%\begin{tcolorbox}[sharp corners, colback=green!30, colframe=green!80!blue, title=Another paragraph with title]
	
	\begin{tcolorbox}[title= Slide \arabic{example} ]
		En la física existen dos teorias con las que se modela la naturaleza, la relatividad general y el modelo estandar. La primera teoria modela la interaccion gravitacional mientras que la segunda hace lo propio con las interacciones de tipo fuerte y electro-debil. 

		La materia oscura es uno de los principales enigmas para la física moderna, y actualmente se cree que es la explicación más probable de los fenómenos gravitatorios que motivaron su postulacion.
		
		Pero por el lado la fisica sub-atomica, el modelo estandar ques es la teoria mas exitosa de la era moderna, no cuenta con un candidato viable de materia oscura.
	
	\stepcounter{example}
	\end{tcolorbox}
	
	\begin{tcolorbox}[title= Slide \arabic{example} ]
	
	A principios de la década de 1930, J.H. Oort descubrió que el movimiento de las estrellas en la Vía Láctea, sugerían la presencia de mucha más masa galáctica que la que nadie había predicho previamente.
	
 	Al estudiar los corrimientos Doppler de estrellas que se mueven cerca al plano de la galaxia, Oort pudo calcular sus velocidades y, por lo tanto, hizo el sorprendente descubrimiento de que las estrellas se mueven lo suficientemente rápido como para escapar de la atracción gravitacional de la masa luminosa en la galaxia.
 	
 	Oort postuló que debe haber más masa presente dentro de la Vía Láctea para mantener estas estrellas en sus órbitas observadas. %Sin embargo, Oort señaló que otra posible explicación era que el 85 \% de la luz del centro galáctico estaba oscurecida por el polvo y materia interviniente o que las mediciones de velocidad de las estrellas en cuestión fueron simplemente erróneas. Más tarde fue corroborado por muchos otros en otras galaxias.
	\stepcounter{example}
	\end{tcolorbox}

	\begin{tcolorbox}[title= Slide \arabic{example}]
	Una de las mas recientes evidencias encontradas, es el llamado "the bullet cluster", que consiste en un sub-cluster "The bullet" colisionando con otro cluster mayor. 
	
	En la colisión de estos dos clusters, las galaxias pasan de largo practicamente sin interaccionar, puesto que están a distancias muy garndes una de otra. Por otra parte la mayor parte de la masa barionica de los clusters, el gas y polvo intergalactico distribuido homogeneamente, interacciona y queda atrapado en el punto donde los clusters se encontraron, calentandose y emitiendo radiación electro-magnética. Ésta radiación fue captada utilizando el telescopio espacial Chandra, y se muestra en la imagen como el color rosa. 
	
	Cuando se calcula la disribución de la masa gravitacional (En Azul) de los clusters utilizando la técnica de lensing, se encuentra que ésta se encuentra concentrada en puntos diferentes que la materia barionica.
	
	\stepcounter{example}
	\end{tcolorbox}

	\begin{tcolorbox}[title= Slide \arabic{example}]
	
	Desde el punto de vista de la astrofisica se tiene que el contenido de masa-energia del universo esta distribuido como se muestra en la figura.
	
	Algo así como un 68\% de Energia oscura, un 27\% de materia oscura y 5\% de materia barionica.
		
	Se puede entonces decir que todo el conocimiento que se tiene sobre fisica fundamental y que llamamos Modelo estándar, habla sobre ese pequeño 5\%. Y aunque es una teoría muy exitosa, el Modelo Estándar es incompleto y no contiene ningun candidato viable a materia oscura. 
	
	
	%Las partículas eléctricamente neutras e débilmente interactuantes en el SM son los neutrinos.
	%pero ¿pueden los neutrinos ser la materia oscura que falta? A pesar de tener la "virtud indiscutible de ser conocido por existir",
	%Hay dos razones principales por las que los neutrinos no pueden dar cuenta de toda la materia oscura del universo.
	
	%-Primero, debido a que los neutrinos son relativistas, un universo dominado por neutrinos habría inhibido la formación de estructuras y causado un "topdown" formación (estructuras más grandes que se forman primero, finalmente se condensan y fragmentan a las que vemos hoy)
	
	%Segundo, WMAP combinado con datos de estructura a gran escala restringe la masa de neutrinos a mv <0.23 eV, que a su vez, hace la densidad cosmológica Si bien los neutrinos representan una pequeña fracción de materia oscura, claramente no pueden ser la única fuente.

	\stepcounter{example}
	\end{tcolorbox}

	
	\begin{tcolorbox}[title= Slide \arabic{example} ]
	De las observaciones se sabe que la materia oscura, no interacciona electrica ni fuertemente, así las interacciones son como mínimo gravitacionales. 

	Si la materia oscura solo se acopla gravitacionalmente a la materia ordinaria, no podria ser detectada en el futuro cercano. Otro posible acercamiento a esete problema es suponer que la materia oscura tiene un acople de tipo debil, y como consecuencia aparecen tres diferentes maneras de detectar materia oscura, las cuales se muestran en la figura.

	La deteccion indirecta es la deteccion de los productos de la aniquilacion de particulas de materia oscura, ejemplos de estas señales son los exesos de rayos gamma, de neutrinos o  de positrones.

	La deteccion directa es cuando particulas de materia oscura perturva la materia del detector, experimentos de este tipo son por ejemplo el Super-Kamiokande, ICECUBE y XENON.
	
	Por ultimo la busqueda en colisionadores, es cuando a partir de particulas del modelo estandar, se producen particulas de materia oscura. Y esta es en la que nosostros nos vamos a centrar.
	\stepcounter{example}
	\end{tcolorbox}

	\begin{tcolorbox}[title= Slide \arabic{example} ]
	Los estudios en aceleradores se llevan a cavo  moticvados por modelos de nueva fisica que orientan la busqueda
	El modelo introduce un fermion vectorial cargado, un escalar neutro y ademas impone una simetria Z2, para asegurar la estabilidad del escalar.
	
	El lagrangiano libre es este.
	
	Con el potencial escalar dado por esta ecuación.
	\stepcounter{example}
	\end{tcolorbox}

	\begin{tcolorbox}[title= Slide \arabic{example} ]%7
	El nuevo termino sinetico, es este
	
	luego el modo de produccion mas diecto en el LHC es el proceso tipo Drell-Yan que se muestra en la figura, donde se crea un par de fermiones vectoriales, y estos decaen a materia oscura y un lepton.
		
	\stepcounter{example}	
	\end{tcolorbox}

	\begin{tcolorbox}[title= Slide \arabic{example} ]
	Para evitar las regiones del espacio de parametros previamente estudiadas y descartadas, nos concentramos en esenarios donde la creacion de materia oscura esta debida exclusivamente su acople con el fermion vectorial.
	
	Tambien se asume que dicho vector no se acopla con el electron, puesto que ya existen cuotas sobre este acople, tanto desde la deteccion indirecta como de colisionadores.
	
	Por ultimo trabajamos en la region de expectro comprimido, que es presisamente la region donde la senal de dos leptones en menos eficiente.
	
	\stepcounter{example}	
	\end{tcolorbox}
	
	\begin{tcolorbox}[title= Slide \arabic{example} ]
	Como se vio en la figura 5, la señal tipica de este modelo es la de dos leptones con cargas electricas opuestas y energía perdida.
	
	Para la region que se describió anteriormente, la probabilidad de que uno o dos leptones no sean detectados es alta, pues debido a que se trabaja en la región de expectro comprimido, la energía disponible para los muones es pequeña.
	
	Además con el fin de tener una topología compatible con los triggers del CMS se requiere al menos un jet adicional. Esto hace la sección efficaz de producción aun mas pequeña.
		\stepcounter{example}	
	\end{tcolorbox}
	
	\begin{tcolorbox}[title= Slide \arabic{example} ] %10
	Esta es la región del espacio de parámetros que nos intereza en este estudio. El area gris está descartada para este modelo por las medidas del Fermi-LAT, el area roja, está permitida, pero la densidad de reliquia del modelo no es suficiente para satisfacer los valores aceptados, y requiere de materia oscura compuesta.
		\stepcounter{example}	
	\end{tcolorbox}
	
	\begin{tcolorbox}[title= Slide \arabic{example} ]
	Los prinsipales background para esta busqueda son: W+Jets, WZ, y la producción de un top. Como se puede ver los Backgrouns tienen entre 3 y 7 ordenes de magnitud mas grandes que la senal, que es de decimas de pico barns. 
	
		\stepcounter{example}	
	\end{tcolorbox}
	
	\begin{tcolorbox}[title= Slide \arabic{example} ]
	Tanto las muestras de montecarlo de senal como las muestras de background se produjeron usando la combinacion de Madgraph de pythia 8 y delphes.
	Donde madgraph se usa para la generacion de eventos, pythia para la hadronizacion y delphes para emular el detector.
		\stepcounter{example}	
	\end{tcolorbox}
	
%	\begin{tcolorbox}[title= Slide \arabic{example} ]
%	Esta es la lista de parametros para Pythia usados para reproducir los resultados experimentales de CMS.
%		\stepcounter{example}	
%	\end{tcolorbox}
	
%	\begin{tcolorbox}[title= Slide \arabic{example} ]
%	En las figuras de la 9 a la 11, muestro algunas variables importantes antes y despues de la sintonizacion de los parametros de pythia.
%		\stepcounter{example}	
%	\end{tcolorbox}
	
%	\begin{tcolorbox}[title= Slide \arabic{example} ]%15
		
%		\stepcounter{example}	
%	\end{tcolorbox}
	
%	\begin{tcolorbox}[title= Slide \arabic{example} ]%16
		
%		\stepcounter{example}	
%	\end{tcolorbox}
	
	\begin{tcolorbox}[title= Slide \arabic{example} ]%17
	Para encontrar una region del espacio sinematico, donde la señal se vea claramente, se sigue una estrategia de cortes maximisando la significancia y la eficiencia de la señal.
	Ambas definidas  en las ecuaciones 2 y tres respectivamente.
	
	Para esto se uso el punto de señal: masa del fermion 145 y delta de masa 10 GeV
		\stepcounter{example}	
	\end{tcolorbox}
	
	\begin{tcolorbox}[title= Slide \arabic{example} ]
		
		\stepcounter{example}	
	\end{tcolorbox}
	
	\begin{tcolorbox}[title= Slide \arabic{example} ]
		
		\stepcounter{example}	
	\end{tcolorbox}
	
	\begin{tcolorbox}[title= Slide \arabic{example} ]
		
		\stepcounter{example}	
	\end{tcolorbox}
	
	\begin{tcolorbox}[title= Slide \arabic{example} ]
		
		\stepcounter{example}	
	\end{tcolorbox}
	
	\begin{tcolorbox}[title= Slide \arabic{example} ]
		
		\stepcounter{example}	
	\end{tcolorbox}
	
	\begin{tcolorbox}[title= Slide \arabic{example} ]
		
		\stepcounter{example}	
	\end{tcolorbox}
	
	\begin{tcolorbox}[title= Slide \arabic{example} ]
		
		\stepcounter{example}	
	\end{tcolorbox}
	
	\begin{tcolorbox}[title= Slide \arabic{example} ]
		
		\stepcounter{example}	
	\end{tcolorbox}
	
	\begin{tcolorbox}[title= Slide \arabic{example} ]
		
		\stepcounter{example}	
	\end{tcolorbox}
	
	\begin{tcolorbox}[title= Slide \arabic{example} ]%27
	En la Tabla se resumen los cortes, con las eficiencias acumuladas y el numero de eventos despues de cortes
		\stepcounter{example}	
	\end{tcolorbox}
	
	\begin{tcolorbox}[title= Slide \arabic{example} ]
	Luego de los cortes calculé la luminocidad nesesaria para tener exclusion. Esto es cual es la luminosidad nesesaria para que en caso de existir la señal esta sea de al menos 3 $\sigma$.
	
	En la figura 21 se muestra para cada punto de masa, la luminosidad nesesaria para alcanzar  3 $\sigma$s. La linea amarilla señala separa la region donde se alcanza 3 $\sigma$s con 3000 fb -1, que es la maxima luminosidad que espera alcanzar el CMS.
		\stepcounter{example}	
	\end{tcolorbox}
	
	\begin{tcolorbox}[title= Slide \arabic{example} ] %29
	Con este estudio encontramos que utilizando este modelo, podemos lograr una esclusión en la zona de espectro comprimido, que usando otro tipo de busqueda digase directa o indirecta no es posible.
		
		
		\stepcounter{example}	
	\end{tcolorbox}
	
	\begin{tcolorbox}[title= Slide \arabic{example} ]
		
		\stepcounter{example}	
	\end{tcolorbox}
	
	\begin{tcolorbox}[title= Slide \arabic{example} ]
		
		\stepcounter{example}	
	\end{tcolorbox}
	
	\begin{tcolorbox}[title= Slide \arabic{example} ]
		
		\stepcounter{example}	
	\end{tcolorbox}
	
	\begin{tcolorbox}[title= Slide \arabic{example} ]
		
		\stepcounter{example}	
	\end{tcolorbox}
	
	\begin{tcolorbox}[title= Slide \arabic{example} ]
		
		\stepcounter{example}	
	\end{tcolorbox}
	
	\begin{tcolorbox}[title= Slide \arabic{example} ]
		
		\stepcounter{example}	
	\end{tcolorbox}
	
	\begin{tcolorbox}[title= Slide \arabic{example} ]
		
		\stepcounter{example}	
	\end{tcolorbox}
	
	
	


\end{document}