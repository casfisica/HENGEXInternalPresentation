 \documentclass[12pt,letterpaper]{article}
 \usepackage[utf8]{inputenc}
 \usepackage{lmodern}
 \usepackage{tcolorbox}
 \usepackage[margin=0.5in]{geometry}
 
 
 \title{Script}
 	
\begin{document}
	%\begin{tcolorbox}[sharp corners, colback=green!30, colframe=green!80!blue, title=Another paragraph with title]
	
	\begin{tcolorbox}[title= Slide 2]
	
	

	In the early 1930s, J.H. Oort found that the motion of stars in the Milky Way hinted at the presence of far more
	galactic mass than anyone had previously predicted. By studying the Doppler shifts of stars moving near the galactic
	plane, Oort was able to calculate their velocities, and thus made the startling discovery that the stars should be
	moving quickly enough to escape the gravitational pull of the luminous mass in the galaxy. Oort postulated that
	there must be more mass present within the Milky Way to hold these stars in their observed orbits. However, Oort
	noted that another possible explanation was that 85\% of the light from the galactic center was obscured by dust and
	intervening matter or that the velocity measurements for the stars in question were simply in error.
	
	Later it was corroborated by many others in other galaxies.
	\end{tcolorbox}

	\begin{tcolorbox}[title= Slide 3]
	
	Recent evidence says that the “smoking-gun” for dark matter comes from the Bullet cluster, that is the result of a subcluster (the “bullet”)
	colliding with the larger galaxy cluster. During the collision, the galaxies within the two clusters passed by each other 
	without interacting. However, the majority of a cluster’s baryonic mass exists in the extremely hot gas between galaxies, and the 
	cluster collision compressed and shock heated this gas; as a result, a huge amount of X-ray radiation was emitted which has 
	been observed by NASA’s Chandra X-ray Observatory. Comparing the location of this radiation (an indication of the location of the 
	majority of the baryonic mass in the clusters) to a mapping of weak gravitational lensing (an indication of the location of the majority
	of the total mass of the clusters) shows an interesting discrepancy; the areas of strong X-ray emission and the largest
	concentrations of mass seen through gravitational lensing are not the same. The majority of the mass in the clusters
	is non-baryonic and gravity “points” back to this missing mass	
	
	\end{tcolorbox}

	\begin{tcolorbox}[title= Slide 3 Motivation]

	Despite its success, the SM does not contain any particle that could act as the dark matter. The only stable,
	electrically neutral, and weakly interacting particles in the SM are the neutrinos. 
	but can the neutrinos be the missing dark matter? Despite having the “undisputed virtue of being known to exist”,
	there are two major reasons why neutrinos cannot account for all of the universe’s dark matter. 
	
	-First, because neutrinos are relativistic, a neutrino-dominated universe would have inhibited structure formation and caused a “topdown” 
	formation (larger structures forming first, eventually condensing and fragmenting to those we see today)
	
	-Second, WMAP combined with large-scale structure data constrains the neutrino mass to mv < 0.23 eV, which
	in turn makes the cosmological density While neutrinos do account for a small fraction of dark 	matter, they clearly cannot be the only source.
	
	\end{tcolorbox}
	
	\begin{tcolorbox}[title= Slide  ]
The scalar-Higgs quartic coupling $ \lambda_{SH}$ is strongly constrained by direct detection experiments~\cite{Akerib:2016vxi},
as well as by the invisible Higgs decay width {[ refs! and add discussion on SSDM]}. %also  ATLAS ref
Adding the new singlet charged fermion $\psi$, odd under $Z_2$, opens up the allowed parameter space for the scalar sector,
with the dominant DM annihilation process at freeze-out being the t and u-channel exchange of a fermion mediator.
We focus on scenarios where
the vector-like portal for DM annihilation is dominant and, therefore, we set initially $ \lambda_{SH}=0$.
Furthermore, we assume that the DM candidate only couples to one single lepton species, the electron, and set $Y_\mu=Y_\tau=0$.
This last assumption is of no consequence on the DM relic density or its direct detection sensitivity,
but affects the decay chain of the next to lightest odd particle (NLOP), and thus its search strategy at colliders.
The quartic coupling $\lambda_S$ is assumed to be small enough, resulting in  a negligible DM self-interaction.
The remaining parameters, $Y_e$,  $m_S$ and  $m_\psi$ are allowed to vary freely.

	\end{tcolorbox}

\end{document}