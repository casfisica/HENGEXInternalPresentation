-Hacer una Validacion con datos del modelo estandar(Utilizando el paper https://cds.cern.ch/record/2204927). L
Cosas que se esperan del SM

-Reproducir, resultados externos con este mismo modelo (arXiv:1801.01846, arXiv:1712.08119)

-De donde sacamos las referencias de xqcut y qcut de los autores de madgraph.

-Tabla de optimizacion con graficas tal y como lo hice para cms.

-Termino cinetico en el lagrangiano. L

-Preguntarle a Diego la tabla de la regiones de parametros de donde salio. L

-Tabla de pesos Fijarme bn que los puce malo.(solo el primero) L

-Tabla de pesos, las XS de donde salieron porque tienen ese valor, cortes cinematicos???

-Poner la linea de 3000fb-1 en la grafica de luminocidades().

-Poner porque este modelo esta relacionado con Fermi-Lat (en la pag 5 de arxiv.org/pdf/0906.3009.pdf y grafica 3)

-Mas espesifico, Modelo sencillo(Porque la escogencia de particulas en especial) 

-Que modelos dan senales similares en cosas supersimetricas(En el paper arXiv:1706.04402v2 ,arXiv:1805.05784).L

-Que es lo que se escluye.

-Como es que se excluye.

-Con que es competitivo.

-En que es novedoso.